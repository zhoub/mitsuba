\subsection{Surface scattering models}
Surface scattering models describe the manner in which light is reflected
by surfaces in the scene, which is fundamentally related to the perceptual
material appearance of an object. To achieve realistic results, Mitsuba 
comes with a library of both general-purpose models (smooth or rough glass, 
metal, plastic, etc.) and specializations to particular materials 
(woven cloth, masks, etc.).

Throughout the documentation and within the scene description language, 
the word \emph{BSDF} is used synonymously with the term ``surface
scattering model''. This is an abbreviation for  \emph{Bidirectional 
Scattering Distribution Function}, a more precise technical 
description of the model's properties. In Mitsuba, BSDFs are 
assigned to \emph{shapes}, which describe the visible surfaces in
the scene. In the scene description language, this assignment can
either be performed by nesting BSDFs within shapes, or they can 
be named and then later referenced by their name.

The following fragment shows an example of both kinds of usages:
\begin{xml}
<scene>
	<!-- Creating a named BSDF for later use -->
	<bsdf type=".. BSDF type .." id="myNamedMaterial">
		<!-- BSDF parameters go here -->
	</bsdf>

	<shape type="sphere">
		<!-- Example of referencing a named material -->

		<ref id="myNamedMaterial"/>
	</shape>

	<shape type="sphere">
		<!-- Example of using an unnamed material -->

		<bsdf type=".. BSDF type ..">
			<!-- BSDF parameters go here -->
		</bsdf>
	</shape>
</scene>
\end{xml}
It is generally more economical to use named BSDFs when they
are used in several places, since this reduces Mitsuba's internal
memory usage.
