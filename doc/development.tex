\part{Development guide}
This chapter and the subsequent ones will provide an overview
of the the coding conventions and general architecture of Mitsuba.
You should only read them if if you wish to interface with the API
in some way (e.g. by developing your own plugins). The coding style
section is only relevant if you plan to submit patches that are meant
to become part of the main codebase.

\section{Code structure}
Mitsuba is split into four basic support libraries:
\begin{itemize}
\item The core library (\code{libcore}) implements basic functionality such as
	cross-platform file and bitmap I/O, data structures, scheduling, as well as logging and plugin management.
\item The rendering library (\code{librender}) contains abstractions
	needed to load and represent scenes containing  light sources, shapes, materials, and participating media.
\item The hardware acceleration library (\code{libhw})
	implements a cross-platform display library, an object-oriented OpenGL
	wrapper, as well as support for rendering interactive previews of scenes.
\item Finally, the bidirectional library (\code{libbidir})
	contains a support layer that is used to implement bidirectional rendering algorithms such as
	Bidirectional Path Tracing and Metropolis Light Transport.
\end{itemize}
A detailed reference of these APIs is available at
\url{http://www.mitsuba-renderer.org/api}. The next sections
present a few basic examples to get familiar with them.

\section{Coding style}
\paragraph{Indentation:} The Mitsuba codebase uses tabs for indentation,
which expand to \emph{four} spaces. Please make sure that you configure your editor
this way, otherwise the source code layout will look garbled.

\paragraph{Placement of braces:} Opening braces should be placed on the
same line to make the best use of vertical space, i.e.
\begin{cpp}
if (x > y) {
	x = y;
}
\end{cpp}

\paragraph{Placement of spaces:} Placement of spaces follows K\&R, e.g.
\begin{cpp}
if (x == y) {
	..
} else if (x > y) {
	..
} else {
	..
}
\end{cpp}
rather than things like this
\begin{cpp}
if ( x==y ){
}
..
\end{cpp}

\paragraph{Name format:} Names are always written in camel-case.
Classes and structures start with a capital letter, whereas member functions
and attributes start with a lower-case letter. Attributes of classes
have the prefix \code{m\_}. Here is an example:
\begin{cpp}
class MyClass {
public:
	MyClass(int value) : m_value(value) { }

	inline void setValue(int value) { m_value = value; }
	inline int getValue() const { return m_value; }
private:
	int m_value;
};
\end{cpp}

\paragraph{Enumerations:} For clarity, both enumerations types and entries
start with a capital \textbf{E}, e.g.
\begin{cpp}
enum ETristate {
	ENo = 0,
	EYes,
	EMaybe
};
\end{cpp}
\paragraph{Constant methods and parameters:} Declare member functions and
their parameters as \code{const} whenever this is possible
and properly conveys the semantics.
\paragraph{Inline methods:} Always inline trivial pieces of code, such
as getters and setters.
\paragraph{Documentation:} Headers files should contain
Doxygen-compatible documentation. It is also a good idea to add
comments to a \code{.cpp} file to explain subtleties of an implemented algorithm.
However, anything pertaining to the API should go into the header file.

\paragraph{Boost:} Use the boost libraries whenever this helps to save
time or write more compact code.

\paragraph{Classes vs structures:}In Mitsuba, classes usually go onto the heap,
whereas structures may be allocated both on the stack and the heap.

Classes that derive from \code{Object} implement a protected virtual
deconstructor, which explicitly prevents them from being allocated on the stack.
The only way they can be deallocated is using the built-in reference
counting. This is done using the \code{ref<>} template, e.g.

\begin{cpp}
if (..) {
	ref<MyClass> instance = new MyClass();
	instance->doSomething()
}   // reference expires, instance will be deallocated
\end{cpp}

\paragraph{Separation of plugins:}Mitsuba encourages that plugins are only
used via the generic interface they implement. You will find that almost all plugins
(e.g. emitters) don't actually provide a header file, hence they can only be accessed
using the generic \code{Emitter} interface they implement. If any kind of special
interaction between plugins is needed, this is usually an indication that the
generic interface should be extended to accomodate this.
