\part{Using Mitsuba}
\textbf{Disclaimer:} This is manual documents the usage, file format, and
internal design of the Mitsuba rendering system. It is currently a work 
in progress, hence some parts may still be incomplete or missing.

\section{About Mitsuba}
Mitsuba is a research-oriented rendering system in the style of PBRT 
(\url{www.pbrt.org}), from which it derives much inspiration.
It is written in portable C++, implements unbiased as well 
as biased techniques, and contains heavy optimizations targeted 
towards current CPU architectures. 
Mitsuba is extremely modular: it consists of a small set of core libraries 
and over 100 different plugins that implement functionality ranging 
from materials and light sources to complete rendering algorithms.

In comparison to other open source renderers, Mitsuba places a strong 
emphasis on experimental rendering techniques, such as path-based 
formulations of Metropolis Light Transport and volumetric
modeling approaches. Thus, it may be of genuine interest to those who
would like to experiment with such techniques that haven't yet found 
their way into mainstream renderers, and it also provides a solid 
foundation for research in this domain.

Other design considerations are are:

\parheader{Performance:}
One important goal of Mitsuba is to provide optimized implementations of the most commonly 
used rendering algorithms. By virtue of running on a shared foundation, comparisons between them can
better highlight the merits and limitations of different approaches. This is in contrast to, say, 
comparing two completely different rendering products, where technical information on the underlying 
implementation is often intentionally not provided.

\parheader{Robustness:}
In many cases, physically-based rendering packages force the user to model scenes with the underlying 
algorithm (specifically: its convergence behavior) in mind. For instance, glass windows are routinely 
replaced with light portals, photons must be manually guided to the relevant parts of a scene, and 
interactions with complex materials are taboo, since they cannot be importance sampled exactly. 
One focus of Mitsuba will be to develop path-space light transport algorithms, which handle such 
cases more gracefully.

\parheader{Scalability:} Mitsuba instances can be merged into large clusters, which transparently distribute and 
jointly execute tasks assigned to them using only node-to-node communcation. It has successfully
scaled to large-scale renderings that involved more than 1000 cores working on a single image.
Most algorithms in Mitsuba  are written using a generic parallelization layer, which can tap 
into this cluster-wide parallelism. The principle is that if any component of the renderer produces
work that takes longer than a second or so, it at least ought to use all of the processing power 
it can get.

The renderer also tries to be very conservative in its use of memory, which allows it to handle 
large scenes (>30 million triangles) and multi-gigabyte heterogeneous volumes on consumer hardware.

\parheader{Realism and accuracy:} Mitsuba comes with a large repository of physically-based
reflectance models for surfaces and participating media. These implementations
are designed so that they can be used to build complex shader networks, while
providing enough flexibility to be compatible with a wide range of different
rendering techniques, including path tracing, photon mapping, hardware-accelerated rendering 
and bidirectional methods.

The unbiased path tracers in Mitsuba are battle-proven and produce 
reference-quality results that can be used for predictive rendering, and to verify 
implementations of other rendering methods. 

\parheader{Usability:}
Mitsuba comes with a graphical user interface to interactively explore scenes. Once a suitable 
viewpoint has been found, it is straightforward to perform renderings using any of the 
implemented rendering techniques, while tweaking their parameters to find the most suitable 
settings. Experimental integration into Blender 2.5 is also available.

\section{License}
Mitsuba is free software and can be redistributed and modified under the terms of the GNU General 
Public License (Version 3) as provided by the Free Software Foundation.


