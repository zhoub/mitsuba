\section{Acknowledgments}
I am indebted to my advisor Steve Marschner for allowing me to devote
a significant amount of my research time to this project. His insightful and 
encouraging suggestions have helped transform this program into much more than
I ever thought it would be.

The architecture of Mitsuba as well as some individual components are based on 
implementations discussed in: \emph{Physically Based Rendering - From Theory 
To Implementation} by Matt Pharr and Greg Humphreys.

Some of the GUI icons were taken from the Humanity icon set by Canonical Ltd.
The material test scene was created by Jonas Pilo, and the environment map
it uses is courtesy of Bernhard Vogl.

The included index of refraction data files for conductors are copied from
PBRT. They are originally from the Luxpop database (\url{www.luxpop.com}) 
and are based on data by Palik et al. \cite{Palik1998Handbook}
and measurements of atomic scattering factors made by the Center For
X-Ray Optics (CXRO) at Berkeley and the Lawrence Livermore National 
Laboratory (LLNL).

The following people have kindly contributed code or bugfixes:
\begin{itemize}
\item Milo\^{s} Ha\^{s}an
\item Tom Kazimiers
\item Marios Papas
\item Edgar Vel\'{a}zquez-Armend\'{a}riz
\item Jirka Vorba
\item Leonhard Gr\"unschlo\ss
\end{itemize}

Mitsuba makes heavy use of the following amazing libraries and tools: 
\begin{itemize}
\item Qt 4 by Digia
\item OpenEXR by Industrial Light \& Magic
\item Xerces-C+\!+ by the Apache Foundation
\item Eigen by Beno\^it Jacob and Ga\"el Guennebaud
\item SSE math functions by Julien Pommier
\item The Boost C+\!+ class library
\item GLEW by Milan Ikits, Marcelo E. Magallon and Lev Povalahev
\item Mersenne Twister by Makoto Matsumoto and Takuji Nishimura
\item Cubature by Steven G. Johnson
\item COLLADA DOM by Sony Computer Entertainment
\item libjpeg-turbo by Darrell Commander and others
\item libpng by Guy Eric Schalnat, Andreas Dilger, Glenn Randers-Pehrson and \mbox{others}
\item libply by Ares Lagae
\item BWToolkit by Brandon Walkin
\item The SCons build system by the SCons Foundation
\end{itemize}
