\newpage
\subsection{Films}
\label{sec:films}
A film defines how conducted measurements are stored and converted into the final
output file that is written to disk at the end of the rendering process. Mitsuba comes with a few
films that can write to high and low dynamic range image formats (OpenEXR, JPEG or PNG), as well
more scientifically oriented data formats (e.g. MATLAB or Mathematica).

In the XML scene description language, a normal film configuration might look as follows
\begin{xml}
<scene version=$\MtsVer$>
	<!-- ... scene contents ... -->

	<sensor type="... sensor type ...">
		<!-- ... sensor parameters ... -->

		<!-- Write to a high dynamic range EXR image -->
		<film type="hdrfilm">
			<!-- Specify the desired resolution (e.g. full HD) -->
			<integer name="width" value="1920"/>
			<integer name="height" value="1080"/>

			<!-- Use a Gaussian reconstruction filter. For
			     details on these, refer to the next subsection -->
			<rfilter type="gaussian"/>
		</film>
	</sensor>
</scene>
\end{xml}
The \code{film} plugin should be instantiated nested inside a \code{sensor} declaration.
Note how the output filename is never specified---it is automatically inferred
from the scene filename and can be manually overridden by passing the configuration parameter
\code{-o} to the \code{mitsuba} executable when rendering from the command line.
