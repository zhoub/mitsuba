\newpage
\subsection{Sensors}
\label{sec:sensors}
In Mitsuba, \emph{sensors}, along with a \emph{film}, are responsible for recording radiance
measurements in some usable format. This includes default choices such as perspective
or orthographic cameras, as well as more specialized sensors that measure the radiance
into a given direction or the irradiance received by a certain surface. This subsection
lists the available choices.

\subsubsection*{Syntax}
In the XML scene description language, a sensor declaration looks as follows
\begin{xml}
<scene version=$\MtsVer$>
    <!-- ... scene contents ... -->

    <sensor type="... sensor type ...">
        <!-- ... sensor parameters ... -->

        <sampler type=" ... sampler type ... ">
            <!-- ... sampler parameters ... -->
        </sampler>

        <film type=" ... film type ... ">
            <!-- ... film parameters ... -->
        </film>
    </sensor>
</scene>
\end{xml}
In other words, the \code{<sensor>} declaration is a child element of the \code{<scene>} (the particular
position in the scene file does not play a role). Nested within the sensor declaration is a
sampler instance (described in \secref{samplers}) and a film instance (described in
\secref{films}).

\subsubsection*{Handedness convention}
Sensors in Mitsuba are \emph{right-handed}.
Any number of rotations and translations can be applied to them
without changing this property. By default they are located at the
origin and oriented in such a way that in the rendered image, $+X$ points left,
$+Y$ points upwards, and $+Z$ points along the viewing direction.

Left-handed sensors are also supported. To switch the handedness,
flip any one of the axes, e.g. by passing a scale transformation like
\code{<scale x="-1"/>} to the sensor's \code{toWorld} parameter.
