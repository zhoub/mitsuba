\section{Basic usage}
\label{sec:basics}
The rendering functionality of Mitsuba can be accessed through
a command line interface and an interactive Qt-based frontend. This section
provides some basic instructions on how to use them.
\subsection{Interactive frontend}
To launch the interactive frontend, run \code{Mitsuba.app} on MacOS, 
\code{mtsgui.exe} on Windows, and \code{mtsgui} on Linux (after sourcing \code{setpath.sh}).
You can also drag and drop scene files onto the application icon or the running program to open them.
A quick video tutorial on using the GUI can be found here: \url{http://vimeo.com/13480342}.
\subsection{Command line interface}
\label{sec:mitsuba}
The \texttt{mitsuba} binary is an alternative non-interactive rendering 
frontend for command-line usage and batch job operation.
To get a listing of the parameters it supports, run
the executable without parameters:
\begin{shell}
$\texttt{\$}$ mitsuba 
\end{shell}
\begin{console}[label=lst:mitsuba-cli,caption=Command line options of the \texttt{mitsuba} binary]
Mitsuba version $\texttt{\MitsubaVersion}$, Copyright (c) $\texttt{\MitsubaYear}$ Wenzel Jakob
Usage: mitsuba [options] <One or more scene XML files>
Options/Arguments:
   -h          Display this help text

   -D key=val  Define a constant, which can referenced as "$\$$key" in the scene

   -o fname    Write the output image to the file denoted by "fname"

   -a p1;p2;.. Add one or more entries to the resource search path

   -p count    Override the detected number of processors. Useful for reducing
               the load or creating scheduling-only nodes in conjunction with
               the -c and -s parameters, e.g. -p 0 -c host1;host2;host3,...

   -q          Quiet mode - do not print any log messages to stdout

   -c hosts    Network rendering: connect to mtssrv instances over a network.
               Requires a semicolon-separated list of host names of the form
                       host.domain[:port] for a direct connection
                 or
                       user@host.domain[:path] for a SSH connection (where
                       "path" denotes the place where Mitsuba is checked
                       out -- by default, "~/mitsuba" is used)

   -s file     Connect to additional Mitsuba servers specified in a file
               with one name per line (same format as in -c)

   -j count    Simultaneously schedule several scenes. Can sometimes accelerate
               rendering when large amounts of processing power are available
               (e.g. when running Mitsuba on a cluster. Default: 1)

   -n name     Assign a node name to this instance (Default: host name)

   -t          Test case mode (see Mitsuba docs for more information)

   -x          Skip rendering of files where output already exists

   -r sec      Write (partial) output images every 'sec' seconds

   -b res      Specify the block resolution used to split images into parallel
               workloads (default: 32). Only applies to some integrators.

   -v          Be more verbose

   -w          Treat warnings as errors

   -z          Disable progress bars

 For documentation, please refer to http://www.mitsuba-renderer.org/docs.html
\end{console}
\lstref{mitsuba-cli} shows the output resulting from this command. The most common
mode of operation is to render a single scene, which is provided as a parameter, e.g.
\begin{shell}
$\texttt{\$}$ mitsuba path-to/my-scene.xml
\end{shell}
It is also possible to connect to network render nodes, which essentially lets Mitsuba parallelize
over additional cores. To do this, pass a semicolon-separated list of machines to 
the \code{-c} parameter. 
\begin{shell}
$\texttt{\$}$ mitsuba -c machine1;machine2;... path-to/my-scene.xml
\end{shell}
There are two different ways in which you can access render nodes:
\begin{itemize}
\item\textbf{Direct}: Here, you create a direct connection to a running \code{mtssrv} instance on
another machine (\code{mtssrv} is the Mitsuba server process). From the the performance 
standpoint, this approach should always be preferred over the SSH method described below when there is 
a choice between them. There are some disadvantages though: first, you need to manually start
\code{mtssrv} on every machine you want to use. 

And perhaps more importantly: the direct communication
protocol makes no provisions for a malicious user on the remote side. It is too costly
to constantly check the communication stream for illegal data sequences, so Mitsuba simply doesn't do it.
The consequence of this is that you should only use the direct communication approach within
trusted networks.

For direct connections, you can specify the remote port as follows:
\begin{shell}
$\texttt{\$}$ mitsuba -c machine:1234 path-to/my-scene.xml
\end{shell}
When no port is explicitly specified, Mitsuba uses default value of 7554. 
\item \textbf{SSH}: 
    This approach works as follows: The renderer creates a SSH connection 
	to the remote side, where it launches a Mitsuba worker instance.
	All subsequent communication then passes through the encrypted link. 
	This is completely secure but slower due to the encryption overhead.
	If you are rendering a complex scene, there is a good chance that it
	won't matter much since most time is spent doing computations rather than
	communicating

	Such an SSH link can be created simply by using a slightly different syntax:
\begin{shell}
$\texttt{\$}$ mitsuba -c username@machine path-to/my-scene.xml
\end{shell}
	The above line assumes that the remote home directory contains
	a Mitsuba source directory named \code{mitsuba},
	which contains the compiled Mitsuba binaries.
	If that is not the case, you need to provide the path to such a directory manually, e.g:
\begin{shell}
$\texttt{\$}$ mitsuba -c username@machine:/opt/mitsuba path-to/my-scene.xml
\end{shell}
	For the SSH connection approach to work, you \emph{must} enable passwordless 
	authentication.
	Try opening a terminal window and running the command \code{ssh username@machine} 
	(replace with the details of your remote connection).
	If you are asked for a password, something is not set up correctly --- please see 
	\url{http://www.debian-administration.org/articles/152} for instructions.

	On Windows, the situation is a bit more difficult since there is no suitable SSH client by
	default. To get SSH connections to work, Mitsuba requires \code{plink.exe} (from PuTTY) to
	be on the path. For passwordless authentication with a Linux/OSX-based
	server, convert your private key to PuTTY's format using \code{puttygen.exe}.
	Afterwards, start \code{pageant.exe} to load and  authenticate the key. All 
	of these binaries are available from the PuTTY website.

	It is possible to mix the two approaches to access some machines directly and others
	over SSH.
\end{itemize}
When doing many network-based renders over the command line, it can become tedious to
specify the connections every time. They can alternatively be loaded from a text file
where each line contains a separate connection description as discussed previously:
\begin{shell}
$\texttt{\$}$ mitsuba -s servers.txt path-to/my-scene.xml
\end{shell}
where \code{servers.txt} e.g. contains
\begin{shell}
user1@machine1.domain.org:/opt/mitsuba
machine2.domain.org
machine3.domain.org:7346
\end{shell}
\subsubsection{Passing parameters}
Any attribute in the XML-based scene description language can be parameterized from the
command line.
For instance, you can render a scene several times with different reflectance values
on a certain material by changing its description to something like 
\begin{xml}
<bsdf type="diffuse">
	<spectrum name="reflectance" value="$\texttt{\$}$reflectance"/>
</bsdf>
\end{xml}
and running Mitsuba as follows:
\begin{shell}
$\texttt{\$}$ mitsuba -Dreflectance=0.1 -o ref_0.1.exr scene.xml 
$\texttt{\$}$ mitsuba -Dreflectance=0.2 -o ref_0.2.exr scene.xml 
$\texttt{\$}$ mitsuba -Dreflectance=0.5 -o ref_0.5.exr scene.xml 
\end{shell}

\subsubsection{Writing partial images to disk}
When doing lengthy command line renders on Linux or OSX, it is possible 
to send a signal to the process using 
\begin{shell}	
$\texttt{\$}$ killall -HUP mitsuba
\end{shell}
This causes the renderer to write out the partially finished 
image, after which it continues rendering. This can sometimes be useful to 
check if everything is working correctly.

\subsubsection{Rendering an animation}
The command line interface is ideally suited for rendering large amounts of files in batch
operation. You can simply pass in the files using a wildcard in the filename. 

If you've already rendered a subset of the frames and you only want to complete the remainder, 
add the \texttt{-x} flag, and all files with existing output will be skipped. You can also 
let the scheduler work on several scenes at once using the \texttt{-j} parameter --- this is 
especially useful when parallelizing over multiple machines: as some of the participating machines
finish rendering the current frame, they can immediately start working on the next one
instead of having to wait for all other cores to finish. Altogether, you
might start the with parameters such as the following
\begin{shell}
$\texttt{\$}$ mitsuba -xj 2 -c machine1;machine2;...  animation/frame_*.xml
\end{shell}
\subsection{Direct connection server}
\label{sec:mtssrv}
A Mitsuba compute node can be created using the \code{mtssrv} executable. By default,
it will listen on port 7554:
\begin{shell}
$\texttt{\$}$ mtssrv
..
maxwell: Listening on port 7554.. Send Ctrl-C or SIGTERM to stop.
\end{shell}
Type \code{mtssrv -h} to see a list of available options.
If you find yourself unable to connect to the server, \code{mtssrv} is probably listening on
the wrong interface. In this case, please specify an explicit IP address or hostname:
\begin{shell}
$\texttt{\$}$ mtssrv -i maxwell.cs.cornell.edu
\end{shell}
As advised in Section~\ref{sec:mitsuba}, it is advised to run \code{mtssrv} \emph{only} in trusted networks.

One nice feature of \code{mtssrv} is that it (like the \code{mitsuba} executable) 
also supports the \code{-c} and \code{-s} parameters, which create connections 
to additional compute servers.
Using this feature, one can create hierarchies of compute nodes. For instance,
the root \code{mttsrv} instance of such a hierarchy could share its work with a 
number of other  machines running \code{mtssrv}, and each of these might also 
share their work with further machines, and so on...

The parallelization over such hierarchies happens transparently---when 
connecting a renderering process to the root node, it sees a machine 
with hundreds or thousands of cores, to which it can submit work without
needing to worry about how exactly it is going to be spread out in 
the hierarchy.

Such hierarchies are mainly useful to reduce communication bottlenecks when distributing
large resources (such as scenes) to remote machines. Imagine the following hypothetical scenario:
you would like to render a 50MB-sized scene while at home, but rendering is too slow. 
You decide to tap into some extra machines available
at your workplace, but this usually doesn't make things much faster because of the relatively slow broadband
connection and the need to transmit your scene to every single compute node involved. 

Using \code{mtssrv}, you can
instead designate a central scheduling node at your workplace, which accepts connections and delegates
rendering tasks to the other machines. In this case, you will only have to transmit the scene once,
and the remaining distribution happens over the fast local network at your workplace.
\subsection{Utility launcher}
\label{sec:mtsutil}
When working on a larger project, one often needs to implement various utility programs that 
perform simple tasks, such as applying a filter to an image or processing
a matrix stored in a file. In a framework like Mitsuba, this unfortunately involves 
a significant coding overhead in initializing the necessary APIs on all supported platforms. 
To reduce this tedious work on the side of the programmer, Mitsuba comes with a utility launcher
called \code{mtsutil}.

The general usage of this command is
\begin{shell}
$\texttt{\$}$ mtsutil [options] <utility name> [arguments]
\end{shell}
For a listing of all supported options and utilities, enter the command without parameters.

\subsubsection{Tonemapper}
\label{sec:tonemapper}
One particularly useful utility that shall be mentioned here is the batch tonemapper, which
loads EXR/RGBE images and writes tonemapped 8-bit PNG/JPGs. This can save much time when one has to
process many high dynamic-range images such as animation frames using the same basic operations, 
e.g. gamma correction, changing the overall brightness, resizing, cropping, etc. The available
command line options are shown in \lstref{tonemap-cli}.

\begin{console}[label=lst:tonemap-cli,caption=Command line options of the \texttt{mtsutil tonemap} utility]
$\texttt{\$}$ mtsutil tonemap
Synopsis: Loads one or more EXR/RGBE images and writes tonemapped 8-bit PNG/JPGs
Usage: mtsutil tonemap [options] <EXR/RGBE file (s)>
Options/Arguments:
   -h             Display this help text

   -g gamma       Specify the gamma value (The default is -1 => sRGB)

   -m multiplier  Multiply the pixel values by 'multiplier' (Default = 1)

   -b r,g,b       Color balance: apply the specified per-channel multipliers

   -c x,y,w,h     Crop: tonemap a given rectangle instead of the entire image

   -s w,h         Resize the output image to the specified resolution

   -r x,y,w,h,i   Add a rectangle at the specified position and intensity, e.g.
                  to make paper figures. The intensity should be in [0, 255].

   -f fmt         Request a certain output format (png/jpg, default:png)

   -a             Require the output image to have an alpha channel

   -p key,burn    Run Reinhard et al.'s photographic tonemapping operator. 'key'
                  between [0, 1] chooses between low and high-key images and
                  'burn' (also [0, 1]) controls how much highlights may burn out

   -x             Temporal coherence mode: activate this flag when tonemapping 
                  frames of an animation using the '-p' option to avoid flicker

   -o file        Save the output with a given filename

   -t             Multithreaded: process several files in parallel

 The operations are ordered as follows: 1. crop, 2. resize, 3. color-balance, 
 4. tonemap, 5. annotate. To simply process a directory full of EXRs in 
 parallel, run the following: 'mtsutil tonemap -t path-to-directory/*.exr'
\end{console}
